\documentclass[11pt]{amsart}

\usepackage{amsmath}
\usepackage[utf8]{inputenc}
\usepackage{physics}

% Letters instead of numbers for subsections
\renewcommand\thesubsection{\Alph{subsection}}

\title[Problem Set 3]{Electrons \\
		\hrulefill \small{ FYS3410: Problem Set 3 } \hrulefill}

\author{Candidate 33}

\date{\today}

\begin{document}

\maketitle

\section{Drude Model}
One of the basic assumptions of the free electron gas (FEG), or Drude model for electrons in solids, is that the mean free path is of the order of the inter-atomic distance. if this  assumption is questionable, it challenges interpretations of all transport properties in terms of FEG, specifically thermal ($\kappa$) and electrical ($\sigma$) conductivities.

The Drude theory makes three assumptions
\begin{enumerate}
\item Electrons have a scattering time $\tau$. The probability of scattering within a time interval $dt$ is $dt/\tau$.
\item Once a scattering event occurs, we assume the electron returns to momentum $\vb{p} = 0$.
\item In between scattering events, the electrons, which are charge $-e$ particles, respond to the externally applied electric field $\vb{E}$ and magnetic field $\vb{B}$.
\end{enumerate}

The failures of the Drude theory:
\begin{itemize}
\item The Hall coefficient frequently is measured to have the wrong sign, indicating a charge carrier with charge opposite to that of the electron.
\item There is no $\frac{3}{2}k_B$ heat capacity per particle measured for electrons in metals. This then makes the Peltier and Seebeck coefficients come out wrong by a factor of $100$.
\end{itemize}

\subsection{Prediction of Wiedemann-Franz}
The most impressive success of the Drude model at the time it was proposed was its explanation of the empirical law of Wiedemann and Franz. 

Firstly, the Lorentz force on an electron is given by
\begin{equation}
\label{eq:lorentz}
\vb{F} = -e(\vb{E} + \vb{v} \times \vb{B})
\end{equation}
Now, consider an electron with momentum $\vb{p}$ at a time $t$. What will the momentum of the electron be at a time $t + dt$? There is a probability $dt/\tau$ that it will scatter to momentum zero. If it does not scatter, it will accelereated as dictated by newtons second law of motion\footnote{$\vb{F} = \frac{d\vb{p}}{dt}$} at probability $1 - dt/\tau$. The expected value of the momentum becomes 
\begin{equation}
\ev{\vb{p}(t + dt)} = \left( 1- \frac{dt}{\tau} \right) (\vb{p}(t) + \vb{F}dt) + \vb{0}\frac{dt}{\tau}.
\end{equation}
This can be rearranged if one keeps the terms in linear order of $dt$
\begin{equation}
\frac{d \ev{\vb{p}}}{dt} = \vb{F} - \frac{\ev{\vb{p}}}{t},
\end{equation}
where $\vb{F}$ is the Lorentz force from equation \ref{eq:lorentz}, of course. Assuming that the electric field is non-zero, the magnetic field is zero, and writing $\ev{\vb{p}}$ as $\vb{p}$ we get
\begin{equation}
\frac{d\vb{p}}{t} = -e\vb{E} -\frac{\vb{p}}{\tau}.
\end{equation}
In a steady state $d\vb{p}/dt= 0$ which gives
\begin{equation}
m\vb{v} = \vb{p} = -e\tau \vb{E} \quad \rightarrow \quad \vb{v} = -\frac{e\vb{E}\tau}{m},
\end{equation}
where $m$ is the mass of the electron and $\vb{v}$ is its velocity.

If there is a density of $n$ of electrons in the metal each with charge $-e$, and they ar all moving at velocity $\vb{v}$, the the electrical current is given by
\begin{equation}
\vb{j} = -en\vb{v} = \frac{e^2\tau n}{m}\vb{E}.
\end{equation}
The electrical conductivity of the metal, defined via $\vb{j} = \sigma\vb{E}$, is given by
\begin{equation}
\label{eq:electricalconductivity}
\sigma = \frac{e^2\tau n}{m}.
\end{equation}

From the kinetic theory of gas we have that the thermal conductivity is
\begin{equation}
\label{eq:kinteticthermal}
\kappa = \frac{1}{3} n C_v \ev{v} \lambda,
\end{equation}
where $\ev{v}$ is the average thermal velocity and  $\lambda = \ev{v} \tau$ is the scattering length. For a monatomic gas the heat capacity per particle is
\begin{equation}
\label{eq:monoheatcap}
C_v = \frac{3}{2}k_B, 
\end{equation}
and 
\begin{equation}
\label{eq:monoavgvel}
\ev{v} = \sqrt{\frac{8k_BT}{\pi m}}.
\end{equation}
Inserting \ref{eq:monoheatcap} and \ref{eq:monoavgvel} into equation \ref{eq:kinteticthermal} yields
\begin{equation}
\kappa = \frac{4}{\pi} \frac{n\tau k_B^2T}{m}.
\end{equation}
While this quantity has the unknown parameter $\tau$ in it, equation \ref{eq:electricalconductivity} contains the same quantity. Thus, we may look at the ratio of thermal conductivity to electrical conductivity,
known as the Lorenz number
\begin{equation}
L = \frac{\kappa}{T\sigma} = \frac{4}{\pi}\left(\frac{k_B}{e} \right)^2.
\end{equation}
A slightly different prediction is obtained by realising that $\ev{v}^2$ was used in this calculation, whereas one can use $\ev{v^2}$ instead\footnote{From kinetic theory: $C_v T = \frac{1}{2}m\ev{v^2}$}. This gives
\begin{equation}
\label{eq:durderesult2}
L = \frac{\kappa}{T\sigma} = \frac{3}{2} \left(\frac{k_B}{e} \right)^2.
\end{equation}

This result was seen as a big success for the Drude model, because it was known for a long time that almost all metals have roughly the same value for this ratio. This fact is called the Wiedemann-Franz law.

\subsection{Free electron Fermi gas}
For a free electron Fermi gas, the heat capacity is given as
\begin{equation}
C_{\text{el}} = \frac{\pi^2}{2} \frac{N k_BT}{T_F}
\end{equation}
and the Fermi velocity is given as
\begin{equation}
v_f = \frac{\hbar k_F}{m},
\end{equation}
where $k_F$ is the Fermi wavevector. This gives a new expression for the heat conductivity
\begin{equation}
\kappa = \frac{\pi^2}{6}\frac{Nnk_BT\hbar^2k_F^2\tau}{T_F m^2},
\end{equation}
while the electrical conductivity ($\sigma$) is the same and given by equation \ref{eq:electricalconductivity}. The Lorenz factor becomes
\begin{equation}
\label{eq:FEFGlorenz1}
L = \frac{\kappa}{T	\sigma} = \frac{\pi^2}{6}\frac{Nk_Bk_F^2\hbar^2}{e^2T_F m},
\end{equation}
which is somewhat of a mess, but replacing the Fermi temperature with
\begin{equation*}
T_F = \frac{E_F}{k_B} = \frac{N\hbar^2k_F^2}{2mk_B}
\end{equation*}
turns the expression in \ref{eq:FEFGlorenz1} into
\begin{equation}
L = \frac{\kappa}{T\sigma} = \frac{\pi^2}{3}\frac{k_B^2}{e^2}.
\end{equation}
This is somewhat similar to what I obtained in equation \ref{eq:durderesult2}.

\end{document}