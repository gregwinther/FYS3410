\documentclass[11pt]{amsart}

\usepackage{amsmath}
\usepackage[utf8]{inputenc}
\usepackage{physics}

% Letters instead of numbers for subsections
\renewcommand\thesubsection{\Alph{subsection}}

\title[Problem Set 3]{Problem Set 3 \\
		\hrulefill \small{ FYS3410: Solid State Physics } \hrulefill}

\author{Candidate 33}

\date{\today}

\begin{document}

\maketitle

\section{Drude Model}
One of the basic assumptions of the free electron gas (FEG), or Drude model for electrons in solids, is that the mean free path is of the order of the inter-atomic distance. if this  assumption is questionable, it challenges interpretations of all transport properties in terms of FEG, specifically thermal ($\kappa$) and electrical ($\sigma$) conductivities.

The Drude theory makes three assumptions
\begin{enumerate}
\item Electrons have a scattering time $\tau$. The probability of scattering within a time interval $dt$ is $dt/\tau$.
\item Once a scattering event occurs, we assume the electron returns to momentum $\vb{p} = 0$.
\item In between scattering events, the electrons, which are charge $-e$ particles, respond to the externally applied electric field $\vb{E}$ and magnetic field $\vb{B}$.
\end{enumerate}
 


\subsection{Prediction of Wiedemann-Franz}
The most impressive success of the Drude model at the time it was proposed was its explanation of the empirical law of Wiedemann and Franz. In a free electron gas model (Drude), the electrons would have kinetic energy equal to the thermal energy


\end{document}